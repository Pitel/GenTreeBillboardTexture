\documentclass[11pt,a4paper]{article}
\usepackage[czech]{babel}
\usepackage[utf8]{inputenc}
\usepackage{times}
\usepackage{url}
\usepackage[textwidth=15.2cm,textheight=23cm]{geometry}
\usepackage{xcolor}

\usepackage{graphicx}

%\usepackage{fancyvrb}
%\DefineVerbatimEnvironment{verbatim}{Verbatim}{}

\usepackage[bf]{caption}

\usepackage[hyperindex,
  plainpages=false,
  pdftex,
  colorlinks,
  pdfborder={0 0 0},
  pdfpagelabels]{hyperref}

\pdfcompresslevel=9

\newcommand{\myincludegraphics}[4]{
  \begin{figure}[!h]
  \centering
  \includegraphics[#1]{#2}
  \caption{#3.} \label{#4}
  \end{figure}
}

% titulní stránka a obsah
\newcommand{\titlepageandcontents}{
  \begin{titlepage}

\vspace*{1cm}

\begin{figure}
  \centering
  \includegraphics[height=6cm]{images/fit.pdf}
\end{figure}

\vspace*{5mm}

\begin{center}
\begin{Large}
Projekt do předmětu PGR -- Počítačová grafika
\end{Large}
\end{center}

\vspace*{5mm}

\begin{center}
\begin{Huge}
Název projektu \\
\end{Huge}
\end{center}

\vspace*{1cm}

\begin{center}
\begin{Large}
\today
\end{Large}
\end{center}

\vfill

\begin{flushleft}
\begin{large}
\begin{tabular}{ll}

\bf Řešitelé:\hspace{3mm} & Jméno Příjmení (\verb_xlogin00@stud.fit.vutbr.cz_) \\
& Jméno Příjmení (\verb_xlogin00@stud.fit.vutbr.cz_) \\
& Jméno Příjmení (\verb_xlogin00@stud.fit.vutbr.cz_) \\
& Fakulta Informačních Technologií \\
& Vysoké Učení Technické v~Brně

\end{tabular}
\end{large}
\end{flushleft}

\end{titlepage}

% vim:set ft=tex expandtab enc=utf8:


  \pagestyle{plain}
  \pagenumbering{roman}
  \setcounter{page}{1}
  %\tableofcontents

  \newpage
  \pagestyle{plain}
  \pagenumbering{arabic}
  \setcounter{page}{1}
}

\def\uv#1{\iflanguage{english}{``#1''}%
                              {\quotedblbase #1\textquotedblleft}}%

% vim:set ft=tex expandtab enc=utf8:


\begin{document}
\titlepageandcontents

%---------------------------------------------------------------------------
\section{Zadání}

%Zde napište informace k zadání (nejde jen o přepis toho, co je na webu;
%komentujte vaše vlastní zpřesnění zadání, zaměření, důrazy, pojetí atd.). Text
%strukturujte, použijte odrážky, číslování$\ldots$
%
%Rozsah: cca 10 odrážek

\begin{itemize}
\item Vytvořit funkci(e), která(é) vygeneruje věrohodnou texturu, kterou půjde použít jako billboard pro zobrazení jednoduchého stromu
\item Funkce bude dobře parametrizovatelná (velikost stromu, barevné ladění, typ stromu, apod.)
\item Funkce nebude pracovat nad globálními daty ani alokovat data - nageneruje data do dané paměti, např: void GenTreeBillboardTexture(char *data, int width, int height, float param, ...);
\item Demonstrujte v jednoduché scéně
\item Důležitá je rychlost funkce, vizuální kvalita výsledku a zapozdření/znovupoužitelnost kódu
\item Popsat algoritmus generování a možnost parametrizace v dokumentaci
\end{itemize}

%---------------------------------------------------------------------------
\section{Použité technologie}

%Zde vypište, jaké technologie vaše řešení používá – co potřebuje k běhu, co
%jste použili při tvorbě, atd. Text strukturujte, použijte odrážky,
%číslování$\ldots$
%
%Rozsah: cca 7 odrážek

\begin{itemize}
\item SDL
\item L-systémy
\item Bresenhamův algoritmus kreslení úseček
\item Git
\item OpenGL
\end{itemize}

%---------------------------------------------------------------------------
\section{Použité zdroje}

%Zde vypište, které zdroje jste použili k tvorbě: hotový kód, hotová data
%(obrázky, modely, $\ldots$), studijní materiály. Pokud vyplyne, že v projektu
%je použit kód nebo data, která nejsou uvedena tady, jedná se o závažný problém
%a projekt bude pravděpodobně hodnocen 0 body.
%
%Rozsah: potřebný počet odrážek

\begin{itemize}
\item TUREK, Michal. Seriál SDL : Hry nejen pro Linux. Root.cz [online]. 2005, [cit. 2010-12-05]. Dostupný z WWW: {\textless}http://www.root.cz/serialy/sdl-hry-nejen-pro-linux/{\textgreater}.
\item Úryvky kódu ze cvičení z předmětu PGR
\end{itemize}

%---------------------------------------------------------------------------
\section{Nejdůležitější dosažené výsledky}

%Popište 3 věci, které jsou na vašem projektu nejlepší. Nejlépe ukažte a
%komentujte obrázky, v nejhorším případě vypište textově.

\begin{itemize}
\item Rozšiřitelnost a variabilita
\item Vektorový přístup
\item Prostorové větve
\end{itemize}

%---------------------------------------------------------------------------
\section{Ovládání vytvořeného programu}

%Stručně popište, jak se program ovládá (nejlépe odrážky rozdělené do
%kategorií). Pokud se ovládání odchyluje od zkratek a způsobů obvykle
%používaných v okýnkových nadstavbách operačních systémů, zdůvodněte, proč se
%tak děje.
%
%Rozsah: potřebný počet odrážek
\subsection{Demo aplikace - užití v praxi}
\begin{itemize}
\item Jedná se o využití textury v 3D prostoru pomocí billboardingu
\item Rozhlížení pohybem myši
\item Pohyb pomocí kláves W, S, A, D
\end{itemize}
\subsection{Aplikace pro parametrizaci stromu}
\begin{itemize}
\item Jedná se o klasickou okenní aplikaci
\item Uživatel si zde může naklikat jednotlivé parametry stromu (typ, detail apod.)
\item Po nastavení parametrů je vygenerován odpovídající obrázek
\end
%---------------------------------------------------------------------------
\section{Zvláštní použité znalosti}

%Uveďte informace, které byly potřeba nad rámec výuky probírané na FIT.
%Vysvětlete je pomocí obrázků, schémat, vzorců apod.
%
%Rozsah: podle potřeby

\subsection{L-systémy}
Pro určení struktury a hlavních rysů stromů jsme použily upravený L-systém,
kde je struktura stromu definována pomocí bezkontextových gramatik. Tyto gramatiky 
obsahují pouze nonterminály a v našem případě jsou přizpůsobeny tak, aby je bylo možné
generovat bez konce (cyklicky do určité hloubky). Jednotlivé nonterminály představují 
určité části stromu (kmen, větev, kmen bez větví apod.).Tyto nonterminály jsou později
parametrizované pro určité typy stromů. Různé variability výsledných stromů jsme
docílili toho, že jsme gramatiku udělali nedeterministickou a vybírali náhodně
odpovídající pravidla pro rozgenerování. Výsledkem této fáze projektu je narozdíl 
od běžných L-systému, kde se výstup ve formě textu předává "želví grafice", 
derivační strom, který je předán další fázi k 3D parametrizaci.

%---------------------------------------------------------------------------
\section{Rozdělení práce v týmu}

%\begin{itemize}
%\item Franta: udělal tohle, udělal tableto, ještě taky toto, vedl tým.
%\item Pepa: pracoval na tom, na tomhle a ještě na tomto.
%\item Mařenka: vytvořila tohle, tamto a ještě něco.
%\end{itemize}
%Pokud to bude vhodné, použijte odrážky místo souvislých vět.
%
%Rozsah: co nejstručnější tak, aby bylo zřejmé, jak byla dělena práce a za co v
%projektu je kdo zodpovědný.

\begin{itemize}
\item Bc. Tomáš Kubík: L-systémy.
\item Bc. Jan Kaláb: rasterizace.
\item Bc. František Skála: parametrizace.
\item Bc. Jan Lipovský: demo.
\end{itemize}

%---------------------------------------------------------------------------
\section{Co bylo nejpracnější}

%Popište, co vám při řešení nejvíce komplikovalo život, s čím jste se museli
%potýkat, co zabralo čas.
%
%Rozsah: 5-10 řádků

\begin{itemize}
\item Git -- až přiliš složitý oproti Subversion.
\item Příroda -- není tak deterministická, jak bychom si přáli.
\item Billboarding v demu -- místama složitější matematika pro představení
\end{itemize}

%---------------------------------------------------------------------------
\section{Zkušenosti získané řešením projektu}

Popište, co jste se řešením projektu naučili. Zahrňte dovednosti obecně
programátorské, věci z oblasti počítačové grafiky, ale i spolupráci v týmu,
hospodaření s časem, atd.

Rozsah: formulujte stručně, uchopte cca 3-5 věcí

%---------------------------------------------------------------------------
\section{Autoevaluace}

Ohodnoťte vaše řešení v jednotlivých kategoriích (0 – nic neuděláno,
zoufalství, 100\% – dokonalost sama). Projekt, který ve finále obdrží plný
počet bodů, může mít složky hodnocené i hodně nízko. Uvedení hodnot blízkých
100\% ve všech nebo mnoha kategoriích může ukazovat na nepochopení problematiky
nebo na snahu kamuflovat slabé stránky projektu. Bodově hodnocena bude i
schopnost vnímat silné a slabé stránky svého řešení.

\paragraph{Technický návrh (50\%):} (analýza, dekompozice problému, volba
vhodných prostředků, $\ldots$) 
Stručně (1-2 řádky) komentujte hodnocení. 

\paragraph{Programování (50\%):} (kvalita a čitelnost kódu, spolehlivost běhu,
obecnost řešení, znovupoužitelnost, $\ldots$)
Stručně (1-2 řádky) komentujte hodnocení. 

\paragraph{Vzhled vytvořeného řešení (50\%):} (uvěřitelnost zobrazení,
estetická kvalita, vhled GUI, $\ldots$)
Stručně (1-2 řádky) komentujte hodnocení. 

\paragraph{Využití zdrojů (50\%):} (využití existujícího kódu a dat, využití
literatury, $\ldots$)
Stručně (1-2 řádky) komentujte hodnocení. 

\paragraph{Hospodaření s časem (50\%):} (rovnoměrné dotažení částí projektu,
míra spěchu, chybějící části řešení, $\ldots$)
Stručně (1-2 řádky) komentujte hodnocení. 

\paragraph{Spolupráce v týmu (50\%):} (komunikace, dodržování dohod, vzájemné
spolehnutí, rovnoměrnost, $\ldots$)
Stručně (1-2 řádky) komentujte hodnocení. 

\paragraph{Celkový dojem (50\%):} (pracnost, získané dovednosti, užitečnost,
volba zadání, cokoliv, $\ldots$)
Stručně (5-10 řádků) komentujte hodnocení. 

%---------------------------------------------------------------------------
\section{Doporučení pro budoucí zadávání projektů}

Co vám vyhovovalo a co nevyhovovalo na organizaci projektů? Které prvky by měly
být zachovány, zesíleny, potlačeny, eliminovány?

%---------------------------------------------------------------------------
\section{Různé}

Ještě něco by v dokumentaci mělo být? Napište to sem! Podle potřeby i založte
novou kapitolu.

\end{document}
% vim:set ft=tex expandtab enc=utf8:
